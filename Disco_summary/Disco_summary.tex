\documentclass{article}
\usepackage{graphicx} % Required for inserting images
\usepackage{amsmath}

\title{DISCO Summary}
\author{Hang Su}
\date{June 2023}

\begin{document}

\maketitle

\section{Solutions to the Disco Primer}
Fluids are defined by spacetime dependent quantites: mass density $\rho$, velocity $v$, and pressure $P$.
In case of the accretion disks of a black hole, we have $5$ real variables: $\rho$, $P$, and $3$
components of $v$. We also have $5$ equations for the $5$ variables along with sonservation laws: 
\begin{enumerate}
    \item Conservation of particle number
    \item Conservation of momentum
    \item Conservation of energy.
\end{enumerate}

These conservation laws give us the equations needed to solve: the Continuity Equation, Euler Equations,
and the Energy Equation.
\begin{eqnarray}
    && \partial_t \rho + \nabla (\cdot \rho v) = 0 \nonumber \\
    && \partial_t (\rho v) + \nabla \cdot (\rho vv + P) =  \rho g \nonumber \\
    && \partial _t (\frac{1}{2} \rho v^2 + e) + \nabla \cdot \left[\left(\frac{1}{2
    } \rho v^2 + e + P \right) v \right] = \rho v \cdot g - \dot q
\end{eqnarray}

The goal for DISCO code is to solve these equations for a hydrodynamic system. Below I have documented
the process of which I preliminarily solve these equations using analytical methods to give rise to 
a simple disk model. 

\subsection{Steady and axissymmetric}

We assume the disk model to be long-lived and axisymmetric. That is, the disk will look the same
from any angle at any moment of time. Thus, we set derivatives of time and angle $\phi$ in cylindrical coordinates
to $0$. Here we will have four ordinary differential equations:

\begin{eqnarray}
    && \textit{Continuity: } \frac{1}{r} \partial_r (r \Sigma V_r) = 0 \nonumber \\
    && \textit{Radial Momentum: } \frac{1}{r} \partial_r \left(r \Sigma 
    V_r^2 + r \Pi - 2r \Sigma \nu \left(\frac{2}{3} \partial_r V_r - 
    \frac{1}{3} \partial_\phi \Omega\right)\right) = 0 \nonumber \\
    && \textit{Angular Momentum: } \frac{1}{r} \partial_r \left(r^3 
    \Sigma V_r \Omega - 2 r \Sigma \nu \left(\frac{1}{2} r^2 \partial_r \Omega\right)\right)
    = 0 \nonumber \\
    && \textit{Energy: } \frac{1}{r} \partial_r \left(r \left(\frac{1}{2} \Sigma v^2
    + \Sigma \epsilon + \Pi\right) V_r - 2r\Sigma \nu(\sigma \cdot v ) r\right)
    = \Sigma v\cdot g - \dot q
\end{eqnarray}

where $\nu$ is viscosity and $\dot q$ is a cooling term, $\Omega $ is angular velocity, 
and
\begin{eqnarray}
    && v \cdot g = -\frac{GM}{r^2} \cdot v \nonumber \\
    && \Sigma = \int \rho dz.
\end{eqnarray} 

\subsection{The razor thin limit}
We are assuming that the disk is razor thin and inviscid. Here we take the approximations of
$H \simeq 0$ and $\nu = 0$. We also set the sounds speed $c_s = 0$, as well as $\Pi$ and $\epsilon$.
Here we find the solutions of  $V_r (r)$ and $\Omega (r)$.

We can do so by taking the insides of the first and third equation as constants, and solve for the other two
equations using results obtained from it. Eventually, we get one of the solutions similar to Saturn's rings: 
$V_r = 0$ and 
\begin{equation}
    \Omega = \sqrt{\frac{GM}{r^3}}.
\end{equation}

As we take the raidal velocity as $0$, we are assuming that the gas stays in a steady orbit without moving inward or
outward. This makes sense in our razor thin and axisymmetric approximation but it is unrealistic since accretion does not
occur with $V_r = 0$.

\subsection{The Accretion Rate and Torque}
We define the accretion rate $\dot M$ as the rate at which matter passes a praticular radius mobing
toward the central object: $\dot M = -2 \pi r \Sigma v_r$. From the continuity eqaution from step 1, we can derive 
\begin{equation}
    \partial_r (r\Sigma V_r) = 0
\end{equation}
and $\dot M = -2 \pi r \Sigma V_r = \textit{constant}$. This means that for a disk that is
steady and axisymmetric, the accretion rate is constant across all radii.
Later, we will use the expression of $V_r$ to solve for other quantities.
\begin{equation}
    V_r = \frac{\dot M}{-2\pi r \Sigma}
\end{equation}

Similarly, we define the torque through the disk $\dot J$ as the rate at which the central object accretes
angular momentum, which is equal to $2\pi$ times the angular momentum flux. 
From the angular momentum equation from step 1, we have
\begin{equation}
    \frac{1}{r} \partial _r (r^3 \Sigma V_r \Omega - r^3 \Sigma \nu \partial_r \Omega) = \partial_r \dot J = 0. 
\end{equation}
Using $V_r$ from above, we get 
\begin{equation}
    \dot J = \dot M r^2 \Omega + 2\pi r^3 \Sigma \nu \partial_r \Omega.
\end{equation}
Rearrange, we have an expression for $\Sigma$
\begin{equation}
    \Sigma = \frac{\dot J - \dot M r^2 \Omega}{2\pi r^3 \nu \partial_r \Omega}.
\end{equation}

\subsection{An accretion disk}

So far, we have two algebraic equations involving $\dot M$ and $\dot J$ and the radial momentum and energy equations. 
It is time to solve the final radial momentum equation assuming the disk is thin $H \ll r$ abd the radial velocity
subsonic $|v_r| < c_s$. The centripital acceleration term should balance the term with the largest other term, in this case, 
$r\Sigma \frac{GM}{r^2}$. This confirms the expression for angular velocity $\Omega = \sqrt{\frac{GM}{r^3}}$.

After simplifying, we obtain the final solution for $\Sigma$
\begin{equation}
    \Sigma = \frac{-1}{3\pi \nu} \left(\frac{\dot J}{\sqrt{GMr}} - \dot M\right),
\end{equation}

and for $V_r$ 
\begin{equation}
    V_r = -\frac{3\nu}{2r}.
\end{equation}

From here, we have obtained a series of constants and expressions that will help the DISCO code solve the hydrodynamic equations.

We define function $f$:
\begin{equation}
    f = 1 - \frac{\dot J}{ \dot M \sqrt{GMr}}.
\end{equation}

Finally, some expressions in terms of $f$:
\begin{eqnarray}
    && \Sigma = \frac{\dot M f }{ 3 \pi \nu} \nonumber \\
    && V_r = \frac{3\nu}{-2rf} \nonumber \\
    && \Omega = \sqrt{\frac{GM}{r^3}}.
\end{eqnarray}


\section{Instructions on using the DISCO code}

Now that we have analytically solved solutions of a simple disk model, it is time that we put these varaibles into the 
disco code. There are many tempates for different simulations performed in disco, but in our case we only focus on
thin accretion disk for singular or binary black holes. 

In order to run the code, we need to provide initial conditions and the instructions for the computer to run the simulations.
I have listed below some of the useful tips for this project running on disco.

The \texttt{Makefile\_opt.in} file contains the initial condition file we will use. We can specify the file name after

\begin{verbatim}
    INITIAL = {FILENAME}.
\end{verbatim}

The file we are putting into the initial section of texttt{Makefile\_opt\.in} is stored in the texttt{initial} directory. 
In this texttt{\.c} file, we can specify the disk model we are using. We can load quantities such as viscocity and adiabatic index into the file using
texttt{setICparams()} function. This is the place we can put the simple disk model we derived above into the simulation.
Specifically, these formulas are in this file:

\begin{eqnarray}
    && f = 1 - \frac{\dot J}{ \dot M \sqrt{GMr}} \nonumber \\
    && \Sigma = \frac{\dot M f }{ 3 \pi \nu} \nonumber \\
    && V_r = \frac{3\nu}{-2rf} \nonumber \\
    && \Omega = \sqrt{\frac{GM}{r^3}}.
\end{eqnarray}



If we want to have a circumbinary black hole simulation, we can add mass profiles in the texttt{planet} directory. 



In the file texttt{in\.par}, we can specify parameters such as the periods of rotation, time step, grid dimensions, hydro parameters etc.
In our practice so far, we only run a small number of checkpoints and observe the simulation through texttt{vdisco}, a visual
platform of which the checkpoints of the simulation can be vidualized. 

In vdisco, the numbers on your keyboard switches among the variables:
\begin{enumerate}
    \item [1] Surface density ($\rho$ or $\Sigma$ in derivations above)
    \item [2] Pressure $P$
    \item [3] Radial velocity $V_r$
    \item [4] Angular velocity $\Omega$
    \item [5] Vertical velocity $V_z$ ($0$ most of the times)
    \item [6] Passive scalar for visualization 
\end{enumerate}

The checkpoints can be scrolled through pressing [ and ] keys. 



\end{document}
