\documentclass{article}
\usepackage{graphicx} % Required for inserting images

\title{Two-Body Problem}
\author{Hang Su}
\date{May 2023}

\begin{document}

\maketitle

\section{Introduction}

The two body problem is an isolated system of two particles which interact through a 
central potential. For two particles with mass $m_1$ and $m_2$, the equations of motion is

\begin{equation} \label{eqn: newt}
    m_1 \ddot{r_1} = F_{21} ; \; m_2 \ddot{r_2} = F_{12}.
\end{equation}

In Newtonian gravity, the potentials are therefore 

\begin{equation}
    U_{12} = u_{21} = \frac{G m_1 m_2}{{|r_1 - r_2|}^2}.
\end{equation}

Since the 2 particles interact via a central potential, the total momentum is conserved. We consider
the center of mass,

\begin{equation}
    R = \frac{m_1r_1 + m_2r_2}{M}
\end{equation}

where $M = m_1 + m_2$.

The time derivative is therefore 

\begin{equation}
    V_{CM} = \frac{m_1v_1 + m_2v_2}{M}.
\end{equation}

Since our system is isolated, the net force and total acceleration is zero, thus $V_{CM}$ is a constant.
The motion of the center of mass is then

\begin{equation}
    R(t) = V_{CM} t.
\end{equation} 

We define the vector of distance between the particles:
\begin{equation}
    r = r_1 - r_2.
\end{equation}
Thus, 
\begin{equation}
    r_1 = R + \frac{m_2 r}{m_1 + m_2}; \; r_2 = R - \frac{m_1 r}{m_1 + m_2}.
\end{equation}

Since we already know $R$, our problem reduces to finding $r$.

If we multiply the masses to \ref{eqn:newt}, we get 
\begin{equation}
    m_1 m_2 \ddot{r_1} = m_2F_{21}; \; m_1m_2 \ddot{r_2} = m_1F_{12}.
\end{equation}

Subtract the second from the first equation and simplify, we get

\begin{equation}
    \frac{m_1m_2}{m_2 + m_1} \ddot{r} = F_{21}
\end{equation}

The force derived from the central potential only depends on distance between the 2 particles.

\begin{equation}
    F_{21} = - \frac{\partial}{\partial r_1} U_{12} (|r_1 - r_2|) =  \nabla U_{12}
\end{equation}

From above, we can summarize them as 

\begin{equation}
    m_\ast \ddot{r} = - \frac{\partial}{\partial r} U(|r|) = F(r)
\end{equation}

where 
\begin{equation}
    m_\ast = \frac{m_1m_2}{m_1 + m_2}
\end{equation}
 is the reduced mass.

\section{Conservation of Angular Momentum}

In this section, we will rewrite the system above in terms of the angular momentum.

\begin{equation}
    L = m_\ast r \times V ;\; V = \dot{r}.
\end{equation}

Since $F$ is in the $r$ direction, the torque has to be $0$, and the angular momentum must be constant.

\begin{equation}
    \frac{d}{dt} L = 0
\end{equation}

Next, we use the \emph{scalar triple product identity},
\begin{equation}
    a \cdot (b\times c) = b \cdot (c\times a) = c \cdot (a\times b)
\end{equation}

and apply it to the angular momentum, we get
\begin{equation}
    r\cdot L = m_\ast r \cdot (r\times V) = m_\ast V \cdot (r\times r) = 0.
\end{equation}

Hence $r$ is always \emph{perpendicular} to angular momentum $L$.

Because the angular momentum is constant, there must be a fixed vector of which $r$ is perpendicular to in space.
Since the position vector is always perpendicular to a certain direction in space, we can reduce the probelm into a 
2D plane in polar coordinates where
\begin{equation}
    r_x = r \cos{\theta}; \; r_y = r\sin{\theta}.
\end{equation}

We choose the $x---y$ plane for the particles to be on in accordance to conventions, and the angular momentum
is oriented to the $z$ direction. The velocity in $\theta$ direction is always positive, therefore the 
particles always rotates around the center of our coordinate system in the same direction.

\begin{equation}
    \dot{\theta} = \frac{|L|}{m_\ast r^2}
\end{equation}

\section{Conservation of Energy}
To begin with, we know that Energy is the sum of kinetic and potential energies.
\begin{equation}
    E = \frac{1}{2}m_\ast v^2 + U(r)
\end{equation}

In polar coordinates, the velocity is
\begin{equation}
    v^2 = \dot{x}^2 + \dot{y}^2 = \dot{r}^2 + r^2 \dot{\theta} ^2.
\end{equation}

The energy conservation becomes
\begin{equation}
    E = \frac{1}{2} m_\ast \dot{r}^2 + \frac{1}{2} m_\ast r^2 \dot{\theta}^2 + U(r).
\end{equation}

\section{Solutions for Two-Body Problem}

Since $\dot{theta} = \frac{L}{mr^2}$, we get the acceleration in $\theta$ direction
\begin{equation}
    \ddot{\theta} = - \frac{\dot{r} L}{mr^3} = -\frac{2\dot{r}\dot{\theta}}{r}.
\end{equation}

Acceleration in $r$ direction is
\begin{equation}
    \ddot{r} = \frac{F}{m} = r \dot{\theta}^2 - \frac{Gm}{r^2}
\end{equation}

because $\ddot{GM/r^2}$ and $r = \frac{x}{\cos{\theta}}$.

Our problem then becomes a system of two second-order ordinary differential equations stated above.
To numerically solve it using RK-4,

we have to use 4 first-order differential equations such as 

\begin{equation}
    V_r = \dot{r}
\end{equation}

\begin{equation}
    V_\theta = \dot{\theta}
\end{equation}

\begin{equation}
    \dot{V_r} = r V_\theta ^2 - \frac{Gm}{r^2}
\end{equation}

\begin{equation}
    \dot{V_\theta} = - \frac{V_r L}{mr^3} 
\end{equation}



\end{document}
