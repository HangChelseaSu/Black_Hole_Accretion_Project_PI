\documentclass{article}
\usepackage{graphicx} % Required for inserting images
\usepackage{amsmath}

\title{Two-Body Problem}
\author{Hang Su}
\date{May 2023}

\begin{document}

\maketitle

\section{Introduction}

The two body problem is an isolated system of two particles which interact through a 
central potential. For two particles with mass $m_1$ and $m_2$, the equations of motion is

\begin{equation} \label{eqn:newt}
    m_1 \ddot{r_1} = F_{21} ; \; m_2 \ddot{r_2} = F_{12}.
\end{equation}

In Newtonian gravity, the potentials are therefore 

\begin{equation}
    U_{12} = U_{21} = \frac{G m_1 m_2}{|r_1 - r_2|}.
\end{equation}

Since the 2 particles interact via a central potential, the total momentum is conserved. We consider
the center of mass,

\begin{equation}
    R = \frac{m_1r_1 + m_2r_2}{M}
\end{equation}

where $M = m_1 + m_2$.

The time derivative is therefore 

\begin{equation}
    V_{CM} = \frac{m_1v_1 + m_2v_2}{M}.
\end{equation}

Since our system is isolated, the net force and total acceleration is zero, thus $V_{CM}$ is a constant.
The motion of the center of mass is then

\begin{equation}
    R(t) = V_{CM} t.
\end{equation} 

We define the vector of distance between the particles:
\begin{equation}
    r = r_1 - r_2.
\end{equation}
Thus, 
\begin{equation}
    r_1 = R + \frac{m_2 r}{m_1 + m_2}; \; r_2 = R - \frac{m_1 r}{m_1 + m_2}.
\end{equation}

Since we already know $R$, our problem reduces to finding $r$.

If we multiply the masses to eq. \ref{eqn:newt}, we get 
\begin{equation}
    m_1 m_2 \ddot{r_1} = m_2F_{21}; \; m_1m_2 \ddot{r_2} = m_1F_{12}.
\end{equation}

Subtract the second from the first equation and simplify, we get

\begin{equation}
    \frac{m_1m_2}{m_2 + m_1} \ddot{r} = F_{21}
\end{equation}

The force derived from the central potential only depends on distance between the 2 particles.

\begin{equation}
    F_{21} = - \frac{\partial}{\partial r_1} U_{12} (|r_1 - r_2|) =  \nabla U_{12}
\end{equation}

From above, we can summarize them as 

\begin{equation}
    m_\ast \ddot{r} = - \frac{\partial}{\partial r} U(|r|) = F(r)
\end{equation}

where 
\begin{equation}
    m_\ast = \frac{m_1m_2}{m_1 + m_2}
\end{equation}
 is the reduced mass.

\section{Conservation of Angular Momentum}

In this section, we will rewrite the system above in terms of the angular momentum.

\begin{equation}
    L = m_\ast r \times V ;\; V = \dot{r}.
\end{equation}

Since $F$ is in the $r$ direction, the torque has to be $0$, and the angular momentum must be constant.

\begin{equation}
    \frac{d}{dt} L = 0
\end{equation}

Next, we use the \emph{scalar triple product identity},
\begin{equation}
    a \cdot (b\times c) = b \cdot (c\times a) = c \cdot (a\times b)
\end{equation}

and apply it to the angular momentum, we get
\begin{equation}
    r\cdot L = m_\ast r \cdot (r\times V) = m_\ast V \cdot (r\times r) = 0.
\end{equation}

Hence $r$ is always \emph{perpendicular} to angular momentum $L$.

Because the angular momentum is constant, there must be a fixed vector of which $r$ is perpendicular to in space.
Since the position vector is always perpendicular to a certain direction in space, we can reduce the probelm into a 
2D plane in polar coordinates where
\begin{equation}
    r_x = r \cos{\theta}; \; r_y = r\sin{\theta}.
\end{equation}

We choose the $x$ - $y$ plane for the particles to be on in accordance to conventions, and the angular momentum
is oriented to the $z$ direction. The velocity in $\theta$ direction is always positive, therefore the 
particles always rotates around the center of our coordinate system in the same direction.

\begin{equation}
    \dot{\theta} = \frac{|L|}{m_\ast r^2}
\end{equation}

\section{Conservation of Energy}
To begin with, we know that Energy is the sum of kinetic and potential energies.
\begin{equation}
    E = \frac{1}{2}m_\ast v^2 + U(r)
\end{equation}

In polar coordinates, the velocity is
\begin{equation}
    v^2 = \dot{x}^2 + \dot{y}^2 = \dot{r}^2 + r^2 \dot{\theta} ^2.
\end{equation}

The energy conservation becomes
\begin{equation}
    E = \frac{1}{2} m_\ast \dot{r}^2 + \frac{1}{2} m_\ast r^2 \dot{\theta}^2 + U(r).
\end{equation}

We call part of this equation the \textbf{effective potential energy}.
\begin{equation}
    U_{\textrm{eff}}(r) = \frac{L^2}{2 m r^2} + U(r)
\end{equation}

Since the angular momentum is constant, in spherical coordinates, 
\begin{equation}
    L = mr^2\dot{\phi} = const; \; \dot{\phi} = \frac{L}{mr^2}
\end{equation}
Here the $\phi$ in spherical coordinates is equivalent to $\theta$ in polar coordinates. 

\section{Solutions for Two-Body Problem (Analytical)}

We have derived that energy of the system is constant
\begin{eqnarray}
    && E = \frac{1}{2}m (\dot{r}^2 + r^2 \dot{\phi}^2) + U(r) = const. \nonumber \\
    && \dot{r} = \frac{dr}{dt} = \sqrt{\frac{2}{m} (E - U(r))\frac{L^2}{m^2 r^2}} \nonumber \\
    && \int \frac{dr}{\sqrt{\frac{2}{m}(E - U(r)) - \frac{L^2}{m^2r^2}}} = \int dt = t .\nonumber
\end{eqnarray}


We want to find trajectory $r(\phi)$.
\begin{eqnarray} \label{eqn:E}
    && \dot{r} = \frac{dr}{dt} = \frac{dr}{d\phi}\frac{d\phi}{dt} = \frac{dr}{d\phi}\frac{L}{mr^2} \nonumber \\
    && E = \frac{1}{2} m \left( \frac{dr}{d\phi}\right)^2 \left( \frac{L}{mr^2}\right)^2 + \frac{L^2}{2mr^2} + U(r) \nonumber \\
    && E = \frac{L}{2m} \left(\frac{dr}{r^2 d\phi}\right)^2 + \frac{L^2}{2mr^2} + U(r) 
\end{eqnarray}

We introduce $b = \frac{1}{r}$ and $db = \frac{-dr}{r^2}$. The equation of energy thus becomes
\begin{equation}
    E = \frac{L^2}{2m} \left[\left(\frac{db}{d\phi}\right)^2 + b^2\right] + U(\frac{1}{b}),
\end{equation}

where if we find $r(\phi)$, and $r(\phi) = \frac{1}{b(\phi)}$, we can find $\phi (t)$. 
We can do this by integrating $\int \dot{\phi} dt = \phi (t)$ where $\dot{\phi} = \frac{L}{mr^2}$.
We then take the derivative of $E$ with respect to $\phi$.

\begin{equation}
    \frac{L^2}{2m} \left[2 \frac{db}{d\phi} \frac{d^2b}{d\phi^2} + 2b \frac{db}{d\phi}\right] + \frac{dU}{db} \frac{db}{d\phi} = 0.
\end{equation}
Divide both sides by $\frac{db}{d\phi}$, assuming $b\neq const.$ and $\frac{db}{d\phi} \neq 0$, we get
\begin{equation}
    \frac{L^2}{m} \left(\frac{d^2b}{d\phi^2} + b\right) = -\frac{du}{db}.
\end{equation}
 Here we obtain \textbf{Binet's equation for central field}.

 \begin{equation}
    \frac{d^2 b}{d\phi^2} + b = - \frac{m}{L^2}\frac{dU}{db}
\end{equation}

\subsection{Kepler Motion}

In a gravitationally bounded two-body system, we take potential 

$U(r) = -\frac{Gm_\ast m}{r} = - \frac{\alpha}{r}$ where $\alpha > 0$ and $\alpha = Gm_\ast m = const$.

Using \textbf{Binet's equation} and $U = -\alpha b$ where $\frac{dU}{db} = -\alpha$,
\begin{equation}
    \frac{d^2 b}{d\phi^2} +b = \frac{m\alpha}{L^2} = \frac{1}{p}; \; p = \frac{L^2}{m\alpha}; \; b = \frac{1+e \cos{\phi}}{p}.
\end{equation}
by the definition of \textbf{orbital eccentricity} $e$.
Since $r = \frac{1}{b}$, we get $r = \frac{p}{1 + e\cos{\phi}}$.

The minimum of $r$ is limited by the cosine function, hence,
\begin{equation}
    r_{\textit{min}} = \frac{p}{1+e}; \; r_{\textit{max}} = \frac{p}{1-e}.
\end{equation}

If we define $a$ as the semi-major axis of the elliptical orbit, we get
\begin{equation}
    \frac{p}{1+e} + \frac{p}{1-e} = 2a; \; p = a(1-e^2).
\end{equation}

\subsection{Energy}
The constant $p$ depends on the orbial eccentricity $e$ and angular momentum $L$. In this section, we show that the energy conservation
can depend on parameter semi-major axis $a$.

Using eqn. \ref{eqn:E} and results obtained above, we can further write energy as 
\begin{equation}
    E = \frac{L^2 (1+e \cos{\phi})^4}{2mp^4} \left[\frac{p^2 e^2 \sin^2{\phi}}{(1+e\cos{\phi})^4}
    + \frac{p^2}{(1+e\cos{\phi})^2}\right] - \frac{\alpha (1+e\cos{\phi})}{p}.
\end{equation}

simplify, we obtain 
\begin{equation}
    E = \frac{(e^2 -1)m\alpha^2}{2a (1-e^2)m\alpha} = -\frac{\alpha}{2a}.
\end{equation}

\subsection{Summary}
From the two conserved quantities, namely energy $E$ and angular momentum $L$, we obtain a solution to the Binet's equation of orbital motion 
for $r$ as a function of $\phi$.
\begin{equation}
    r = \frac{p}{1+e \cos{\phi}}
\end{equation}
where $p = \frac{L^2}{m\alpha}$ and $E = -\frac{\alpha}{2a}$. Hence different $L$ and $E$ will give rise to different shapes of orbits.

Using Newtonian gravity where the kinetic energy and gravitational potential are equal to each other,
\begin{equation}
    \frac{1}{2} m v_e^2 = \frac{GmM}{r},
\end{equation}
we obtain the equation for escape velocity of a system.

\begin{equation}
    v_e = \sqrt{\frac{2Gm}{r}}.
\end{equation}

In case of gravitationally bounded objects, the total energy is equal to kinetic energy of the object orbiting the total mass
minus the potential energy.
\begin{equation}
    E_{\textit{tot}} = \frac{1}{2}mv^2 - \frac{GMm}{r}.
\end{equation}

\begin{enumerate}
    \item If $E <0 $, the orbit is a bounded elliptical orbit.
    \item If $E = 0$, the orbit is parabolic where it is marginally unbounded. It comes in from infinitely far away and orbit through
    and heads back to infinitely away.
    \item If $E > 0$, the orbit is hyperbolic where the object does not orbit.
\end{enumerate}

\subsubsection{Energy \& Momentum as a function of $a$ and $e$}
From above results, we can write the energy and momentum in terms of semi-major axis and eccentricity.
\begin{equation}
    E = -\frac{\alpha}{2a}
\end{equation}

\begin{equation}
    L =  \sqrt{(1-e^2)am\alpha}
\end{equation}

In terms of the two masses $m_1$ and $m_2$,
\begin{equation}
    E = -\frac{Gm_1 m_2}{a}
\end{equation}
\begin{equation}
    L =  m_1 m_2 \sqrt{(1-e^2) \frac{Ga}{m_1+m_2}}
\end{equation}

\subsubsection{$a$ and $e$ as a function of energy \& momentum}
We can hence do the inverse of above equations to show that eccentricity and semi-major axis can be written as a function of energy and angular momentum.

\begin{equation}
    a = - \frac{\alpha}{2E}
\end{equation}
\begin{equation}
    e = \sqrt{1 + \frac{2 E L^2}{m \alpha^2}}
\end{equation}

In terms of $m_1$ and $m_2$,

\begin{equation}
    a = - \frac{Gm_1 m_2}{2E}
\end{equation}
\begin{equation}
    e = \sqrt{1 + \frac{2E L^2 (m_1 + m_2)}{m_1^3 m_2^3G^2}}
\end{equation}

The above results limit the conserved energy to be negative to have a bounded elliptical orbit.

\subsubsection{Time derivatives of $a$ and $e$}
In the more realistic situations, the eccentricity and semi-major axis of the orbit can change over time. We can hence write the time derivatives of $a$ and $e$ as a function of individual masses, 
angular momentum, and energy.

\begin{equation}
    \dot{a} = -\frac{G}{2} \left[\frac{\dot m_1 m_2 + m_1 \dot m_2}{E} - \frac{m_1m_2 \dot E}{E^2}\right]
    = a \left(\frac{\dot m_1}{m_1} + \frac{\dot m_2}{m_2} - \frac{\dot E}{E}\right)
\end{equation}

where $\dot E = \dot E_1 + \dot E_2$. The time derivative of energy is the sum of time derivatives of individual energies.


For eccentricity,

\begin{equation}
    \begin{split}
        \dot e 
        & = \frac{1}{G^2 m_1^6 m_2^6 e}  \cdot \\
        &  \left(m_1^3 m_2^3 EL^2 (\dot m_1 + \dot m_2) + (m_1 + m_2)(\dot E L^2 - 2 E L \dot L) \right.\\
        & \left. - EL^2(m_1 + m_2) (3 m_1^2 \dot m_1 m_2^3+ 3m_2^2 \dot m_2 m_1^3) \right)
    \end{split}
\end{equation}


% \begin{equation} \label{eq1}
%     \begin{split}
%     \dot e & =  \frac{1}{G} \left[(m_1m_2)^{-3/2} J (-2E(m_1m_2))^{-1/2} (\dot E (m_1 + m_2) + E(\dot m_1 + \dot m_2))\\
%      & = \frac{1}{2} \pi r^2
%     \end{split}
%     \end{equation}

\section{Solutions for Two-Body Problem (Numerical)}

Since $\dot{\theta} = \frac{L}{mr^2}$, we get the acceleration in $\theta$ direction
\begin{equation}
    \ddot{\theta} = - \frac{\dot{r} L}{mr^3} = -\frac{2\dot{r}\dot{\theta}}{r}.
\end{equation}

Acceleration in $r$ direction is
\begin{equation}
    \ddot{r} = \frac{F}{m} = r \dot{\theta}^2 - \frac{Gm}{r^2}
\end{equation}

because $\ddot{GM/r^2}$ and $r = \frac{x}{\cos{\theta}}$.

Our problem then becomes a system of two second-order ordinary differential equations stated above.
To numerically solve it using RK-4,

we have to use 4 first-order differential equations such as 

\begin{equation}
    V_r = \dot{r}
\end{equation}

\begin{equation}
    V_\theta = \dot{\theta}
\end{equation}

\begin{equation}
    \dot{V_r} = r V_\theta ^2 - \frac{Gm}{r^2}
\end{equation}

\begin{equation}
    \dot{V_\theta} = - \frac{2V_r L}{mr^3} 
\end{equation}



\end{document}
