\documentclass{article}
\usepackage{graphicx} % Required for inserting images
\usepackage{amsmath}

\title{Accretion Power in Astrophysics Notes}
\author{Hang Su}
\date{June 2023}

\begin{document}

\maketitle

\section{Chapter 1: Accretion as a source of energy}

\textbf{Eddington limit}: The maxinum luminosity of a given mass created by the accretion rate
being controlled by the outward momentum transferred from the radiation to the accreting material
by scattering and absorption.

\begin{equation}
    L_{\textit{EDD}} = 4 \pi G M m_p c / \sigma T
\end{equation}
where $m_p$ is proton mass. 

At greater luminosities the outward pressure of radiation would exceed the inward gravitational attraction and accretion
would be halted.

For stars with a given mass-luminosity relation this argument yields a maximum stable mass. 

\textbf{Standard candles}: stars whose typical luminosities are close to their Eddington limit.

\textbf{Accretion luminosity}: For accretion powered objects the Eddington limit implies a limit on the steady
accretion rate, $\dot{M} (g s^{-1})$. If all the kinetic energy of infalling matter is given up to radiation at the 
stellar surface, $R_\ast$, then the accretion luminosity is
\begin{equation}
    L_{\textit{acc}} =  GM\dot{M}/ r_\ast
\end{equation}

For black holes, since the radius does not refer to a hard surface but only to a region into which
matter can fall and from which it cannot escape, much of the accretion energy could disapperar into 
the hole and simply add to its mass, rather than being radiated.

The uncertainty in this case can be pramaetrized by the introduction of a dimensionless quantity $\eta$,
the \emph{efficiency}.
\begin{equation}
    L_{textit{acc}} = 2 \eta GM\dot{M} / R_\ast = \eta \dot{M} c^2
\end{equation}
(From Schwarzschild radius)

We see that $\eta = 0.007$ for the burning of hydrogen to helium.

At the limit where the material accreting on to a black hole could be lowered into the hole 
infinitesimally slowly, all the rest mass energy could, in principle, be extracted and we should have
$\eta = 1$. 

Interestingly, from $\eta$, despite its extra conpactness, a stellar mass black hole may be no more
efficient in the conversion of gravitational potential energy to radiation than a neutron start of similar masses.

\vbox{}

\textbf{Note:} Only massive black holes are plausible candidates for accreting objects in active galactic nuclei.



\section{Chapter 2: Gas dynamics}


\section{Chapter 4: Accretion in binary systems}


\section{Chapter 5: Accretion discs}







\end{document}
